Land Cover classification is a fundamental research topic with applications geography,
ecology, geology, forestry, land policy and planning etc... 
With that in mind the focus of this work is to introduce the reader to the usage of Synthetic Aperture Radars (SAR) 
and its applications to Remote Sensing and Land Cover Classification. 

This work will focus on the usage of SAR data to the detection of forest and non forest areas and change target detection. 
Since Forest preservation is something crucial for the environment preservation nowadays it is very important to have technology 
that can perceive fast changes in the scenario of forest, specially if these changes are due to illegal deforestation. 
Change Target detection methods are also very useful to identify rapid changes in a scene and it is of great importance in remote sensing, monitoring environmental changes and land use. 

This work will provide useful information on the current state of the art methods of creating Landcover 
maps from SAR data and change target detection using SAR data. It will be also be shown how the textural information can improve these methods.

The work presented here is part of a work that was developed at the German Aerospace Center (DLR) 
during the year of 2019. The rest of the work was done as a Master Student at ITA during the years of 2020 and 2021.

The work developed at DLR was done under the supervision of Paola Rizzoli and Andrea Pulella 
during the time I was a exchange student there (from february 2019 until november 2019). 
After that period the work developed at DLR was used for this master thesis 
(which was written under the supervision of ITA Professor Marcelo Pinho).