asdf Land Cover classification is a fundamental research topic with applications geography,
ecology, geology, forestry, land policy and planning etc... With that in mind the focus of this work is to introduce the reader to the usage of Synthetic Aperture Radars (SAR) and its applications to Remote Sensing and Land Cover Classification. 

This work will focus mainly on the usage of SAR data to the detection of forest and non forest areas. Since Forest preservation is something crucial for the environment preservation nowadays it is very important to have technology that can perceive fast changes in the scenario of forest, specially if these changes are due to illegal deforestation. 

This work will provide useful information on the current state of the art methods of generating these Forest maps from SAR data and will describe a new method for using temporal information between SAR Images acquisitions for improving even further these classification maps.

The work presented here is part of a work that was developed at the German Aerospace Center (DLR) during the year of 2019. The rest of the work is related to combining the temporal information with the spacial information of SAR images to create forest maps.

The work developed at DLR was done under the supervision of Paola Rizzoli and Andrea Pulella during the time I was a exchange student there (from february 2019 until november 2019). After that period the work developed at DLR was used for this bachelor thesis (which was written under the supervision of ITA Professor Marcelo Pinho) and was continued to be further researched with the intent of becoming a Master Thesis (again under the supervision of Marcelo Pinho). 