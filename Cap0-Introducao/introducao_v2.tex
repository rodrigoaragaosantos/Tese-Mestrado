Deforestation is the removal of forests or standing trees from land
with the purpose of converting it to non-forest use. Deforestation can be used to convert land to farms,
crops or urban use and includes the conversion of forests with purposes of agriculture, fuel extraction,
construction or manufacturing. Human made deforestation is a phenomenon common to all forests in the world, but
the most common occurrence of deforestation is in tropical forests \cite{Alina} - according to a 2017 study from University of Maryland,
tropical forests loss an area of 158,000 square kilometers in 2017, an area the size of Bangladesh \cite{maryland}.

Currently, approximately 38\% of Earth's land surface is covered by forests \cite{WWF}, this is only two thirds of existing
forest coverage before the expansion of agriculture in the last century\cite{WID}. The United Nations Framework Convention on Climate Change
(UNF CCC) says that the major cause of deforestation is due to agriculture. Subsistence farming is responsible for 49\% of deforestation; commercial agriculture
is responsible for 33\%; logging is responsible for 13\% and fuel extraction is responsible for 6\% \cite{UNFCCC}.

Deforestation has significant impact on the Earth's environment, therefore, it can also have significant impact on human life currently living on the planet.
The most common environmental effects of deforestation are: atmospheric, hydrological, soil, and biodiversity.

The most troublesome atmospheric effects of deforestation are related to global warming since deforestation is not only a contributor to global warming, but
one of the major causes of greenhouse effect. Tropical deforestation alone is responsible for over 20\% of world greenhouse emissions \cite{Chirac}.

The hydrological effects are also concerning since the water cycle is heavily affected by deforestation. Trees are responsible for extracting water from the soil
and releasing it into the atmosphere, therefore, when a forest is removed, the atmosphere receives much less water, and a drier climate ensues. Deforestation also reduces soil cohesion,
so that erosion, flooding and landslides become more frequent \cite{Rogge}.

Deforestation can also cause reduction in biodiversity, since a degraded environment can impact the life of different species \cite{umich}. Since
tropical forests are the most diverse ecosystem in Earth, about 80\% of Earth's ecosystem can be found there, and since the most common place of deforestation
is in tropical forests, then the majority of Earth's biodiversity is at risk\cite{Mogato}.

Besides environmental effects, deforestation can have significant economical effects. According to the World Economic Forum (WEF), half of the global Gross Domestic Product (GDP)
is dependent upon nature, and for every dollar invested in nature restoration, there is a long term profit of 9 dollars. Therefore, environmental damages could halve the living standard of society.
According to 2008's Convention on Biological Diversity (CBD), deforestation impacts could diminish world's GDP by up to 7\% in the year of 2050.

Due to these reasons, deforestation problems are one of the biggest concerns of the 21st century, something that countries are interest to deal with. Since deforestation in tropical forests is mostly related to illegal
logging (which can be hard to detect) we try to propose a solution that can improve the monitoring and detection of illegal deforestation.

\section{Objective}

If one is trying to detect when, and where, illegal deforestation is happening, one must understand the problems that come with it. There are multiple problems that must be dealt with, the first one being: which approach will be used for
monitoring this illegal deforestation. On this thesis, it was chosen to use Remote Sensing techniques to aid on the solution of this problem.

According to \cite{Schott1996RemoteST} remote sensing is "the acquisition of information about an object or phenomenon without making physical contact with the object, in contrast to in situ or on-site observation". Currently, remote sensing
refers to the use of satellite or aircraft-based sensors to detect and classify objects on earth. The problem of detecting and classifying objects and areas on earth is called Land Cover Classification.

Land Cover classification is a fundamental research topic with applications geography,
ecology, geology, forestry, land policy and planning etc... With this in mind, the focus of this work is to use Synthetic Aperture Radar (SAR) techniques to present
new techniques Land Cover Classification and Target detection.

This work will focus on the usage of SAR data to the detection of forest and non
forest areas and change target detection. Since Forest preservation is something crucial
for the environment preservation nowadays it is very important to have technology that
can perceive fast changes in the scenario of forest, specially if these changes are due to
illegal deforestation. Change Target detection methods are also very useful to identify
rapid changes in a scene, and it is of great importance in remote sensing, monitoring
environmental changes and land use.

When trying to detect illegal deforestation, one can use several approaches to solve the problem. The first approach, and the most naive one, is to create a land cover map, update it regularly and compare differences to identify
areas where deforestation is happening. There are some drawbacks to this method: creating land cover classification maps is a very time-consuming task, and the overall accuracy of the maps is not high enough to provide a robust detection for small changes in forest coverage \cite{Rodrigo}.(which means that it is hard to detect when small areas were deforested).

Another method uses a more creative approach to illegal deforestation detection: Since man-made objects are easier to detect using SAR data \cite{manmade}, one can try to detect man-made objects used for deforestation, e.g., vehicles used for deforestation.
One advantage of this method is that vehicles are made of metallic materials, which possess high reflectivity, therefore, can be easily detected using SARs. One disadvantage of this method is that vehicles are much smaller — compared to forest areas — so the detection of these small objects will not always be easier.

On this work both approaches will be developed and validated. The work will consist of two parts: The first part will be focused on the creation of land cover classification Maps
using SAR data, and was developed in the German Aerospace Center (DLR), in the year of 2019, under the supervision of Dr. Paola Rizzoli and Andrea Pulella. The second part of the work will work on the
change detection of vehicles hidden under tree foliage using SAR data, and was developed at ITA, during the years of 2020 and 2021, under the supervision of Dr. Marcelo Pinho and Dr. Renato Machado.

\section{Document Organization}
The remainder of this dissertation is organized as follows:

\begin{enumerate}
    \item Chapter 2 - Bibliographic Review: this chapter presents a bibliographic review of SAR, a review of methods of using SAR data for the creation
          of land cover classification maps, and a review of SAR methods for the problem of change target detection.
    \item Chapter 3 - Methods and Materials: this chapter presents the methods proposed for the creation of the land cover classification maps, the methods proposed for the
          task of change detection, and also presents a review about textural information (a key concept used in this thesis for the solution of both problems). This chapter also presents
          the datasets used for both problems: the Sentinel-1 \& TANDEM-X dataset over the Amazon Rainforest (used for the Land Cover Classification problem), and the CARABAS-II dataset (used for the change detection problem).
    \item Chapter 4 - Analysis of Texture Results: this chapter presents the results of the textural information extraction of the datasets and shows why it is relevant to the solution of the problem
    \item Chapter 5 - Classification Results of the Land Coverage Map: this chapter will present the classification results of the creation of the land coverage map using Sentinel-1 SAR data and the textural information that was previously extracted.
    \item Chapter 6 - Results of the Change Detection Algorithm: this chapter will present the results of the proposed change detection algorithm using SAR data using CARABAS-II SAR Data and the textural information that was previously extracted.
    \item Chapter 7 - Conclusion: This chapter will summarize the dissertation results and propose further improvements that could be developed in the future.
\end{enumerate}

