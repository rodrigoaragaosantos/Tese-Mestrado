\section{Summary}
In this chapter we will use the textural developed in previous chapters 
to create a change detection algorithm (CDA). The CDA will be assessed
using the CARABAS-II image dataset, which is a set of images acquired by a Very High Frequency (VHF) Ultra Wideband (UWB) SAR system.
The CDA is based on the UNET convolutional neural network (CNN) architecture and will use the 
textura information as additional inputs for image classification.
It will be demonstrated that the 
use of textural information can improve the overall performance of the algorithm
in terms of probability of detection and false alarm rate. 

\section{Wavelength Resolution SAR Systems}
Wavelength resolution SAR images are radar systems with resolution in the order of 
the radar system Wavelength. According to (ref), the resolution of a cell of a SAR image can be
calculated by:

\begin{equation}
    \delta = \frac{\lambda_c c}{4 \theta_H B}
\end{equation}

where $\lambda_c$ is the wavelength corresponding to the radar central frequency, $\theta_H$
is the aperture angle (in radians), $c$ is the speed of light and $B$ is the system bandwidth.
According to (17ref), Wavelength SAR systems are not sensitive to small scatterers inside the resolution cell,
thus the scattering process is mainly due to scatterers with dimensions in the order of the system wavelength.
Since the resolution of the cell is similar to the resolution of the scatterer, there can be only one single scatterer inside each cell, 
therefore the image will not be greatly affected by speckle noise.

Due to all that, according to (63ref), low resolution SAR systems are better suited to Foliage Penetrating (FOPEN) applications.
Also according to (63ref), VHF-band is the optimal radar system for FOPEN applications for vehicle-sized target detection.

\section{The Carabas-II Dataset}
