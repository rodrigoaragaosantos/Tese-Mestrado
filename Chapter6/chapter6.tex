\section{Summary}
In this chapter we will use the textural developed in previous chapters 
to create a change detection algorithm (CDA). The CDA will be assessed
using the CARABAS-II image dataset, which is a set of images acquired by a Very High Frequency (VHF) Ultra Wideband (UWB) SAR system.
The CDA is based on the UNET convolutional neural network (CNN) architecture and will use the 
textura information as additional inputs for image classification.
It will be demonstrated that the 
use of textural information can improve the overall performance of the algorithm
in terms of probability of detection and false alarm rate. 

\section{Wavelength Resolution SAR Systems}
Wavelength resolution SAR images are radar systems with resolution in the order of 
the radar system Wavelength. According to (ref), the resolution of a cell of a SAR image can be
calculated by:

\begin{equation}
    \delta = \frac{\lambda_c c}{4 \theta_H B}
\end{equation}

where $\lambda_c$ is the wavelength corresponding to the radar central frequency, $\theta_H$
is the aperture angle (in radians), $c$ is the speed of light and $B$ is the system bandwidth.
According to (17ref), Wavelength SAR systems are not sensitive to small scatterers inside the resolution cell,
thus the scattering process is mainly due to scatterers with dimensions in the order of the system wavelength.
Since the resolution of the cell is similar to the resolution of the scatterer, there can be only one single scatterer inside each cell, 
therefore the image will not be greatly affected by speckle noise.

Due to all that, according to (63ref), low resolution SAR systems are better suited to Foliage Penetrating (FOPEN) applications.
Also according to (63ref), VHF-band is the optimal radar system for FOPEN applications for vehicle-sized target detection.

\section{The Carabas-II System}

The CARABAS-II is the second generation SAR system designed by the Swedish Defense Research Agency (FOI) for FOPEN applications.
The CARABAS-II has participated in numerous military campaigns and has been used for reasearch purpose since the nineties.
The radar is a VLF UHB SAR system that transmits HH-polarized radio waves in the frequency range of 
20-90 MHz, therefore having resolution in the range of 3.3-15 m.

In the table below the system parameters for the CARABAS-II system are presented for the
flight campaigns that were used in this work. 

\begin{table}[h]
    \centering
    \begin{tabular}{|c|c|}
        \hline
        System Parameters & Values \\ \hline
        Nominal flight altitude & 3 - 9 km \\ \hline
        Nominal flight speed & 127 m/s \\ \hline
        Frequency band & 20-86 MHz \\ \hline
        Aperture angle & 90 degrees \\ \hline
        Transmitted power & 500 W \\ \hline
        Pulse modulation & Non-linear frequency modulation \\ \hline
        Radio Frequency Interference (RFI) sniff & On \\ \hline
        Frequency sub-bands & 35 (36 with RFI-sniff \\ \hline
        Frequency step & 1.875 Mhz \\ \hline
        Center frequencies & 21.25-85Mhz \\ \hline
        Pulse repetition & frequency 5000Hz \\ \hline
        Pulse length & 15$\mu$s \\ \hline
        Maximum range & 26.4 km \\ \hline
    \end{tabular}
    \caption{CARABAS-II SAR system parameters. Source: (76 ref)}
    \label{tab:carabas_system}
\end{table}

\section{The CARABAS-II Dataset}

By trying to promote research on change detection algorithms for wavelength-resolution
images, FOI created a dataset of images acquired by CARABAS-II and made it publicly available. 
This dataset is the one used to test and assess the quality of the proposed CDA.

The dataset consists of 24 SAR images selected from over 150 images obtained during different flight campaigns.
Each image covers the same ground area of 6 $km^2$ (3 km vertically and 2 km horizontally)
and is given in the form of a 3000 X 2000 matrix, where each pixel size is 1km x 1km.
According to (ref 77,62,78) the images are already calibrated, pre-processed and geocoded.

The location of the image dataset is inside the military base station Missile Test Area North
Vidsel in northern Sweden in 2002. The test site is a region near the village of Nausta (ref 75).
The vegetation of the area is dominated by Scots Pine tree (ref 76), which consists of small and medium size trees.
According to (ref75) the area also contains fields, roads and lakes.

With the objective of testing CDA quality, it was deployed 25 testing targets over the testing with different configuration 
and arrangements. The testing targets consists of ten TGB11 model military vehicles, eight TGB30 model 
military vehicles, and seven TGGB40 model military vehicles. The dimensions of each vehicle are presented in the table (numero).
From the table (numero) it can be seen that the dimensions of the vehicles are similar to the wavelength of the CARABAS-II system,
therefore all advantages of targets with similar dimension to the wavelength previously mentioned hold true for the dataset.

\begin{table}[h]
    \centering
    \begin{tabular}{|c|c|c|c|c|}
        \hline
        Military Vehicle & Lenght & Width & Height & Quantity \\ \hline
        TGB11 & 4.4 m & 1.9 m & 2.2 m & 10 \\ \hline
        TGB30 & 6.8m & 2.5m & 3.0 m & 8 \\ \hline
        TGB40 & 7.8m & 2.5m & 3.0m & 7 \\ \hline
    \end{tabular}
    \caption{Target dimensions}
    \label{tab:vehicle_dimensions}
\end{table}


The dataset of 24 images were acquired using four different flight passes. Each 
flight pass has an incidence angle of 58 degrees, used the StripSAR mode and was acquired with the radar looking left (ref 75,76).
The vehicles were positioned in two diferent forests, Forest 2 being in the northwest of the field, and forest 1 being in the
southeast of the test area. 

\begin{figure}[h]
    \centering
    \includegraphics{chapter6/carabas_vehicles_fisico.jpg}
    \caption{Military vehicles used as target. Left:TGB11. Middle:TGB30. Right:TGB40. 
    This picture also depicts the vegetation characteristics of the test area}
    \label{fig:veiculos}
\end{figure}

The vehicles in mission 2 are positioned in forest 2 and have a heading angle of 225 degrees pointing southwest direction;
vehicles in mission 3 are positioned in forest 2 and have a heading angle of 315 degrees pointing northwest direction;
vehicles in mission 4 are located in forest 1 and have the same heading angle of mission 2;
vehicles in mission 5 are located in forest 1 and have a heading angle of 270 degrees pointing west direction.
Vehicles in mission 2 and 3 are separated approximately by 50m, as such for vehicles in mission 4 and 5.
Figure \ref{fig:carabas_vehicles} presents images of each mission with the vehicles area highlighted in red.

\begin{figure}[h]
    \centering
    \includegraphics{chapter6/carabas_vehicles.jpg}
    \caption{Examples of CARABAS-II images where (a) is mission 2, (b) is mission 3, (c) is mission 4 and (d)
    is mission 5}
    \label{fig:carabas_vehicles}
\end{figure}

In table \ref{tab:flight_mission} it is presented the summary with the information of each image in the dataset

\begin{table}[h]
    \centering
    \begin{tabular}{|c|c|c|c|c|}
        \hline
        Image Number & Mission & Pass & Flight Heading (degrees) & Target Heading \\ \hline
        1 & 2 & 1 & 225 & 225  \\ \hline
        2 & 2 & 2 & 135 & 225  \\ \hline
        3 & 2 & 3 & 225 & 225  \\ \hline
        4 & 2 & 4 & 135 & 225 \\ \hline
        5 & 2 & 5 & 230 & 225 \\ \hline
        6 & 2 & 6 & 230 & 225 \\ \hline
        7 & 3 & 1 & 225 & 315 \\ \hline
        8 & 3 & 2 & 135 & 315 \\ \hline
        9 & 3 & 3 & 225 & 315 \\ \hline
        10 & 3 & 4 & 135 & 315 \\ \hline
        11 & 3 & 4 & 135 & 315 \\ \hline
        12& 3 & 6 & 230 & 315  \\ \hline
        13 & 4 & 1 & 225 & 225  \\ \hline
        14 & 4 & 2 & 135 & 225  \\ \hline
        15 & 4 & 3 & 225 & 225  \\ \hline
        16 & 4 & 4 & 135 & 225  \\ \hline
        17 & 4 & 5 & 230 & 225  \\ \hline
        18 & 4 & 6 & 230& 225 \\ \hline
        19 & 5 & 1 & 225 & 270  \\ \hline
        20 & 5 & 2 & 135 & 270  \\ \hline
        21 & 5 & 3 & 225 & 270  \\ \hline
        22 & 5 & 4 & 135 & 270  \\ \hline
        23 & 5 & 5 & 230 & 270  \\ \hline
        24 & 5 & 6 & 230 & 270  \\ \hline
    \end{tabular}
    \caption{Measurements parameters for each image}
    \label{tab:flight_mission}
\end{table}

\section{Traditional Change Detecion in Wavelength-Resolution SAR Images}

The field of FOPEN target detection using wavelength-resolution SAR images (both in VHF and UHF bands) has been 
around of decades (ref 25,48,79,80). Among those studies, the CARABAS-II system dataset was used in many different FOPEN studies 
(ref). 

As previously mentioned, the objective of the study is to create a change detection method that will compare 
different images of the same target field (in this case, the forests in Sweden) and try to identify the targets that 
have changed position in the image (in this case, the military vehicles TGB11, TGB30 and TGB40 ) and give the exact location of the vehicle positions.


There are several change detection methods that were already used in the CARABAS-II dataset.
The first method that yielded accuracy high enough for real world applications was based in Bayes linear classification (ref81).
After that, a myriad of new methods were proposed, such as: Likelihood-ratio test (LRT) (ref 61), combination of LRT and Space-Time 
adaptative processing (STAP) (ref 76) among others. Traditionally CDAs were also mainly based on traditional statistical decision theory, e.g., 
traditional hypothesis testing criterion methods, such as maximum a posteriori 
criterion (ref), likelihood ratio test(ref), generalized likelihood ratio test (ref), 
or Bayesian theory approaches (ref). Several of those CDAs have achieved high accuracy 
in terms of true positives, but most show unsatisfactory performance in terms of false positives percentage (ref).





