Given the information presented here on this work on SARs and land cover classification, the objective of this graduation thesis is to study techniques on classification maps generation and information extraction based on modelling the temporal decorrelation of SAR images, focusing specifically on the SENTINEL-1 satellite. The data for this SAR was provided by DLR.

With the objective of improving land cover classification, different methods for temporal decorrelation were studied and analysed in partnership with DLR, until a final method was chosen and used throughout the rest of the work. 

Chapter one was the background information on SAR principles and coherence estimation. On this chapter it was presented the main bibliography on SAR and coherence estimation and the problem was briefly explained. On chapter two the problem on temporal decorrelation modelling was explained with more depth and it was explained how using these decorrelation models could be useful for improving land cover classification techniques. On chapter three it was presented not only the methods used for tackling the problem but also how to implement it, including the pipelines important for the data processing.

The programming tools chosen for solving the problem and validating it were Python and C++. These tools were chosen based on versatility, how easy they are to implement using these languages and computational time for the programs.