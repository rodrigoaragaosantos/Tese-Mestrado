O foco deste trabalho é investigar algumas aplicações de imagens de SAR, mais especificamente, aplicações para classificação de cobertura de terra e sensoriamento remoto.
O trabalho apresentará modelos de última geração para a criação de mapas de cobertura florestal e criará seu próprio mapa de cobertura florestal para a Floresta Amazônica.
O trabalho também introduzirá modelos de última geração para o problema da detecção de alvos, concentrando-se na detecção de veículos escondidos sob a folhagem das árvores, e proporá um algoritmo de detecção de alvos de mudança.

Também será demonstrado que esses métodos podem ser muito melhorados combinando as informações das imagens SAR com informações adicionais ocultas - informações de texturas - que podem ser extraídas das imagens.
Usando métodos de processamento estatístico de sinais e usando essa informação oculta como input adicional para algoritmos de aprendizagem de máquinas para classificação, tais como Random Forest e Convolutional Neural Network. 

O trabalho foi validado, no caso da classificação da cobertura terrestre, sobre a Floresta Amazônica, utilizando pilhas interferométricas de banda-C do sistema Sentinel-1 e pilhas interferométricas de banda-X do sistema TANDEM-X que foram fornecidas pelo DLR.
No caso da detecção de alvos, o algoritmo foi validado sobre a área florestal da Suécia, utilizando dados do sistema SAR de banda-VHF CARABAS-II .

Os resultados para a classificação da cobertura terrestre mostraram precisão de mais de 98\% - para os dados TANDEM-X - e mais de 92\% - para os dados Sentinel-1 - para o mapa de cobertura florestal sobre a Floresta Amazônica.
O algoritmo de detecção de mudança proposto, usando o sistema CARABAS-II, mostrou uma precisão de detecção superior a 97\% e mostrou ao mesmo tempo que a taxa de Falso Alarme pode ser grandemente reduzida aproveitando a informação das texturas que foram extraída.
