O foco desta dissertação de mestrado é investigar algumas aplicações fazendo-se uso de imagens de SAR, mais especificamente, aplicações para classificação de cobertura de terra e detecção de mudança.
Como primeira contribuição, o trabalho apresenta modelos para a criação de mapas de cobertura florestal e criará seu próprio mapa de cobertura florestal para a Floresta Amazônica. Essa dissertação de mestrado considera dois métodos de estração de atributos para o propósito de classificação, a saber, i) o método da matriz de coocorrência de níveis de cinza (GLCM, \textit{grey level co-occurrence matrix}), e ii) o método de soma e diferença de histogramas (SD, \textit{sum and difference histogram}).
A segunda contibuição dessa dissertação apresenta uma nova abordagem para o problema de detecção de alvos, baseado em detecção de mudança, concentrando-se na detecção de veículos escondidos sob a folhagem de árvores. O método de detecção de alvo proposto combina informação de amplitude com informação de texturas (informação oculta)  das imagens SAR. O método explora recursos de processamento de sinais e de inteligência artificial.  Para fins de validação dos métodos propostos, consideraram-se imagens três bancos de dados SAR. Para o problema de classificação de cobertura terrestre, utilizaram-se dois bancos de dados de imagens sobre a Floresta Amazônica. O primeiro banco de dados consiste em uma pilha de imagens interferométricas na banda-C, obtidas com o  sistema Sentinel-1 e o segundo banco de dados, em uma pilha de imagens interferométricas na banda-X, obitidas com o sistema TANDEM-X. Para o problema de detecção de alvos, o algoritmo foi validado utilzando um banco de dados de imagens de amplitude, banda VHF, obtidas com o sistema SAR CARABAS-II. Como resultados principais, obtevê-se uma precisão da classificação da cobertura terrestre de aproximadamente 98\% para os dados TANDEM-X e 92\% para os dados Sentinel-1. Em relação ao algoritmo de detecção de mudanca proposto, obtevê-se uma precisão de detecção de aproximadamente 97\% para uma taxa de falso alarme de 0.0034 erros por metro quadrado.