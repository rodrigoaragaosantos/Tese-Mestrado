%O foco deste trabalho é investigar a melhor forma de combinar imagens interferométricas de SAR (InSAR) para a classificação da cobertura terrestre. A idéia proposta aproveita dados multi-temporais, adquiridos em curtos intervalos de observação (séries de tempo curto). Imagens que foram adquiridas com uma linha de base temporal maior devem ter uma maior perda de coerência interferométrica, enquanto imagens adquiridas com uma linha de base temporal menor devem ter uma correlação maior entre elas. A idéia principal do trabalho é aproveitar o fato de que diferentes alvos no solo irão descorrelacionar temporalmente a diferentes taxas. Modelando a evolução temporal da descorrelação temporal é possível extrair numericamente parâmetros que podem ajudar a classificar a cena. Ao combinar esses parâmetros com outros parâmetros da cena, como \textit{backscatter}, é possível utilizá-los como entradas para algoritmos de \textit{Machine Learning} para classificação, como Redes Neurais ou \textit{Random Forest} (neste trabalho foi escolhida a \textit{Random Forest} por ter rendido um resultado melhor). O trabalho foi validado no caso da classificação da cobertura terrestre na Europa e na floresta Amazônica, utilizando pilhas interferométricas de banda C do Sentinel-1, que foi fornecido pelo DLR. Foram consideradas três classes distintas para este trabalho: áreas florestadas, áreas não florestadas e superfícies artificiais.
%Comparando o resultado com o mapa de ocupação do solo CORINE de 2012, os resultados mostram um nível de concordância de 91\% e mais de 85\% para a floresta Amazônica.
