The focus of this master thesis is to investigate some applications using SAR imagery, more specifically, applications for land cover classification and change detection.
As a first contribution, the work presents models for the creation of forest cover maps and will create its own forest cover map for the Amazon Rainforest. This master's dissertation considers two attribute stratification methods for the purpose of classification, namely, i) the gray level co-occurrence matrix method (GLCM, \textit{grey level co-occurrence matrix}), and ii) the histogram sum and difference method (SD, \textit{sum and difference histogram}).
The second contribution of this dissertation presents a new approach to the target detection problem based on change detection, focusing on the detection of vehicles hidden under tree foliage. The proposed target detection method combines amplitude information with texture information (hidden information) from SAR images. The method exploits signal processing and artificial intelligence features.  For the purpose of validation of the proposed methods, images three SAR databases were considered. For the land cover classification problem, two databases of images over the Amazon Rainforest were used. The first database consists of a stack of C-band interferometric images obtained with the Sentinel-1 system and the second database consists of a stack of X-band interferometric images obtained with the TANDEM-X system. For the target detection problem, the algorithm was validated using a database of amplitude images, VHF band, obtained with the SAR CARABAS-II system. As main results, a land cover classification accuracy of about 98\% for the TANDEM-X data and 92\% for the Sentinel-1 data is obtained. Regarding the proposed change detection algorithm, a detection accuracy of approximately 97\% is obtained for a false alarm rate of 0.0034 errors per square meter.