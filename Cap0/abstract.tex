The focus of this work is to investigate some applications of SAR Images, more specifically, applications to land cover classification and remote sensing.
The work will introduce state of the art models for the creation of forest coverage maps and will create its own forest coverage map for the Amazon Rainforest.
The work will also introduce state of the art models for the problem of target detection, focusing in detecting hidden vehicles under tree foliage, and will propose an change target detection algorithm.

It will also be demonstrated that those methods can be greatly improved by combining the information of the SAR images with additional hidden information - textural information - that can be extracted from the images
using statistical signal processing methods and using it as additional input for Machine Learning algorithms for classification, such as Random Forests and Convolutional Neural Networks. 

The work was validated , on the case of land cover classification, over the Amazon Rainforest, using Sentinel-1 C-band interferometric stacks and TANDEM-X X-band interferometric stacks which were provided by DLR.
On the case of target detection, the algorithm was validated over the Swedish forest area, using data from the VHF-band wavelenth-resolution SAR System CARABAS-II.

The results for the land cover classification showed accuracy of over 98\% - for the TANDEM-X data -  and over 92\% - for the Sentinel-1 data - for the forest coverage map over the Amazon Rainforest.
The proposed change detection algorithm, using the CARABAS-II system, showed detection accuracy of over 97\% and showed at the same time  that the False Alarm rate can be greatly reducing by taking advantage of the textural information that was extracted.
