%The focus of this work is to investigate how to best combine interferometric SAR (InSAR) images for land cover classification. The idea proposed takes advantage of multi-temporal data, acquired over short observation intervals (short-time-series). Images which were acquired with a larger temporal baseline are expected to have a larger interferometric coherence loss, while images acquired with a smaller temporal baseline are expected to have a higher correlation between them. The main idea of the work is to take advantage of the fact that different targets on the ground will temporally decorrelate at different rates. By modelling the temporal evolution of the temporal decorrelation it is possible to extract parameters numerically that can help classify the scene. By combining these parameters with other parameters of the scene, like backscatter, it is possible to use it as inputs for Machine Learning algorithms for classification, like Neural Networks or Random Forest (on this work Random Forest was chosen because it yielded a better result). The work was validated on the case of land cover classification over Europe(CORINE) and over the Amazon Rainforest, using Sentinel-1 C-band interferometric stacks which was provided by DLR. It was considered three different classes for this work: forested areas, non-forested areas and artificial surfaces.
%By comparing the result with the CORINE land cover map of 2012 the results show a level of agreement of 91\% and over 85\% for the Amazon Rainforest.